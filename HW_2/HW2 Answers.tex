\documentclass[a4paper, 10pt]{article}

\usepackage[english]{babel}
\usepackage[T1]{fontenc}
\usepackage[utf8]{inputenc}
\usepackage{textcomp}
\setlength{\marginparwidth}{2cm}

\usepackage{comment}
\usepackage{todonotes}

\usepackage{amsmath}
\usepackage{amssymb}
\usepackage{latexsym}
\usepackage{bm}

\usepackage{enumitem}
\usepackage{array}
\setlength\extrarowheight{5pt}

\usepackage{xcolor}
\usepackage{graphicx}
\graphicspath{ {./img/} }

\newcommand\scalemath[2]{\scalebox{#1}{\mbox{\ensuremath{\displaystyle #2}}}}

\usepackage{hyperref}
\usepackage{listings}
\usepackage{color}
\definecolor{dkgreen}{rgb}{0,0.6,0}
\definecolor{gray}{rgb}{0.5,0.5,0.5}
\definecolor{mauve}{rgb}{0.58,0,0.82}
\lstset{frame=tb,
    language=Python,
    aboveskip=3mm,
    belowskip=3mm,
    showstringspaces=false,
    columns=flexible,
    basicstyle={\small\ttfamily},
    numbers=none,
    numberstyle=\tiny\color{gray},
    keywordstyle=\color{blue},
    commentstyle=\color{dkgreen},
    stringstyle=\color{mauve},
    breaklines=true,
    breakatwhitespace=true,
    tabsize=3
}

\title{Homework Assignment N°2}
\author{AML3\\Thibault Douzon\\Georgios Lioutas}
\date{December 15th, 2018}

\begin{document}
\maketitle

\pagebreak

\tableofcontents

\pagebreak
\section{Exercise 1}
\subsection{Part a}
We want to compute for each sample in the data the responsability 
$\gamma(z_k)$ of each Bernoulli distribution.
\\
What we call responsability is the probability knowing the data occured that model $z_k$ is involved.
We can rewrite it in term of probabilities:
$$
\gamma(z_k) = P(z_k \vert x) = \frac{p(x \vert z_k) p(z_k)}{p(x)}
$$
Where $p(z_k)$ is the prior of each model, $p(x\vert z_k)$ is given by the distribution $z_k$ (here Bernoulli)
and $p(x)$ is the sum over all models inthe mixture of the probability they produced $x$.
\\
For example with the first point:
\\
\begin{itemize}
    \item The prior is given by our assumptions: $p(z_k) = \left[0.5, 0.5 \right]$,
    
    \item Each model is a Bernoulli distribution, \\hence $p(x\vert z_k) = 
    \left(
        \begin{array}{c}
            x_0+x_1 \\
            x_0
        \end{array}
    \right) {\mu_k}^{x_0}(1-\mu_k)^{x_1}$
    
    \item And the total probability of $x$ is just the marginalization over all models: $p(x) = \sum_{z_k} p(x \vert z_k) p(z_k)$
\end{itemize}
We finally obtain the following values:
$$
p((1,4)) \approx 0.243
$$
and the responsabilities:
$$
\gamma(z_1) = 0.678
$$
$$
\gamma(z_2) = 0.322
$$
We repeat the same procedure for all given inputs and obtain the following results:
\begin{center}
    \begin{tabular}{ |c|c|c| }
        \hline
        $x$ & $\gamma(z_{n1})$ & $\gamma(z_{n2})$ \\
        \hline
        $(1, 4)$ & 0.678 & 0.322 \\
        \hline
        $(3, 2)$ & 0.345 & 0.655 \\
        \hline
        $(4, 1)$ & 0.208 & 0.792 \\
        \hline
        $(2, 3)$ & 0.513 & 0.487\\
        \hline
    \end{tabular}
\end{center}

\subsection{Part b}
We now want to improve our mixture of model in order to better explain the data we saw.
We must compute a new value for the prior and the parameter of each Bernoulli distribution.
\\
Using the responsabities we just computed in the previous step, the EM algorithm gives us 
formulas to updates thoses values:
\\
The new prior of model $z_k$:
$$
\pi^2_k = \frac{\gamma(z_k)}{\sum_i \gamma(z_i)}
$$
The new proportion of head of model $z_k$:
$$
\mu^2_k = \frac{\gamma(z_k)\mu_k}{\sum_i \gamma(z_i)\mu_i}
$$
But because we now operate in batch after seeing multiple data, we need to modify
those equations by summing over all seen points:
$$
\pi^2_k = \frac{\sum_n\gamma(z_{nk})}{\sum_n \sum_i \gamma(z_{ni})}
$$
$$
\mu^2_k = \frac{\sum_n \gamma(z_{nk})\mu_k}{\sum_n \sum_i \gamma(z_{ni})\mu_i}
$$
With the responsabilities from the previous exercise and the parameters given,
we obtain the following results:
$$
\pi^2 = \left[0.436, 0.564\right]
$$
$$
\mu^2 = \left[0.409, 0.570\right]
$$
This means that we have two models with priors $\pi^2$. Each one of them is still a
Bernoulli model. The first one give a probability of 0.409 to H and the second a probability
of 0.570 to H.

\section{Exercise 2}
\subsection{Part a}
We are now working with Hidden Makov Models. To compute $\alpha$ and $\beta$,
we must do a forward pass and then a backward pass.
\\
Starting with $\alpha$ and the forward pass:
\\
The general formula for $\alpha$ is the folowing:
$\alpha(z_n) = p(x_1, \dots, x_n, z_n)$.
\\
But it can be computed with this other recursive definition:
$$
\alpha(z_1) = p(x_1, z_1) = p(x_1\vert z_1)p(z_1) 
$$
$$
\alpha(z_n) = p(x_n\vert z_n)\sum_{z_{n-1}} \alpha(z_{n-1})p(z_n\vert z_{n-1}) 
$$
Where the likelihoods $p(x\vert z)$ are given by the matrix $\phi$ and the 
transition probabilities $p(z_n \vert z_{n-1})$ are given by the matrix A

For $t=1$, we simply have to multiply the prior of each state by the emission probability
that corresponds to the data:
$$
\alpha(z_1) = \left[0.5\cdot 0.7, 0.5\cdot 0.2\right] 
$$
For the next iteration, we now must use the second part of the recusrsive definition.
\\
For the sake of simplicity, let's only compute the first model value of $\alpha$:
$$
\alpha(z_{2,1}) = p(x_2\vert z_{2,1}) \sum_{z_1} \alpha(z_1)p(z_{2,1}\vert z_1) 
$$
Because the second character in the observation is T, $p(x_2\vert z_{2,1}) = 0.7$.
$$
\alpha(z_{2,1}) = 0.7 (0.35 \cdot 0.6 + 0.1 \cdot 0.4) = 0.14
$$
We apply this for both models and each time step and we obtain the following table:
\begin{center}
    \begin{tabular}{ |c|c|c|c|c| }
        \hline
         & $t=1$ & $t=2$ & $t=3$ & $t=4$ \\
        \hline
        $\alpha_{t,1}$ & 0.35 & 0.14 & 0.026 & 0.018 \\
        \hline
        $\alpha_{t,2}$ & 0.1 & 0.05 & 0.083 & 0.039\\
        \hline
    \end{tabular}
\end{center}

Now we deal with the backward pass and compute $\beta$.
\\
The general formula for $\beta$ is the folowing:
$\beta(z_n) = p(x_{n+1}, \dots, x_N \vert z_n)$.
\\
Again we can use a recursive definition to compute it:
$$
\beta(z_N) = 1
$$
$$
\beta(z_n) = \sum_{z_{n+1}} \beta(z_{n+1})p(x_{n+1}\vert z_{n+1})p(z_{n+1}\vert z_n)
$$
For example we can compute the value of $\beta$ for the first model at $t=3$:
$$
\beta(z_{3,1}) = \sum_{z_4} \beta(z_4)p(x_4\vert z_4)p(z_4\vert z_{3,1})
$$
$$
\beta(z_{3,1}) = 1\cdot 0.3 \cdot 0.4 + 1\cdot 0.8 \cdot 0.6 = 0.6
$$
By applying the same procedure for the whole observation, we get the following:
\begin{center}
    \begin{tabular}{ |c|c|c|c|c| }
        \hline
         & $t=1$ & $t=2$ & $t=3$ & $t=4$ \\
        \hline
        $\beta_{t,1}$ & 0.120 & 0.312 & 0.6 & 1 \\
        \hline
        $\beta{t,2}$ & 0.152 & 0.268 & 0.5 & 1\\
        \hline
    \end{tabular}
\end{center}

\subsection{Part b}
Once we have computed $\alpha$ and $\beta$ for all positions in the observation, it is easy to compute the overall probability
of the observation:
$$
p(x_n) = \sum_{z_n} \alpha(z_n)\beta(z_n)
$$
And because $\beta(z_N) = 1$ we get
$$
p(x_N) = \sum_{z_N} \alpha(z_N)
$$
In our case:
$$
P(O) = p(x_N) = 0.018 + 0.039 = 0.057
$$
This is the probability that this observation has been generated by one of the models in the mixture.

\subsection{Part c}
We can use the formulas from the book to update the prior, the probability of H for each model 
and the transition matrix:
\\
Our new prior is the following:
$$
\pi = \left[0.774, 0.226\right]
$$
The new transition matrix is:
$$
\begin{bmatrix}
    0.505 & 0.495\\
    0.627 & 0.374
\end{bmatrix}
$$
And finally the new $\mu$ is:
$$
\mu = \left[0.736, 0.234\right]
$$
\end{document}
