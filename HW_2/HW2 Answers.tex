\documentclass[a4paper, 10pt]{article}

\usepackage[english]{babel}
\usepackage[T1]{fontenc}
\usepackage[utf8]{inputenc}
\usepackage{textcomp}
\setlength{\marginparwidth}{2cm}

\usepackage{comment}
\usepackage{todonotes}

\usepackage{amsmath}
\usepackage{amssymb}
\usepackage{latexsym}
\usepackage{bm}

\usepackage{enumitem}
\usepackage{array}
\setlength\extrarowheight{5pt}

\usepackage{xcolor}
\usepackage{graphicx}
\graphicspath{ {./img/} }

\newcommand\scalemath[2]{\scalebox{#1}{\mbox{\ensuremath{\displaystyle #2}}}}

\usepackage{hyperref}
\usepackage{listings}
\usepackage{color}
\definecolor{dkgreen}{rgb}{0,0.6,0}
\definecolor{gray}{rgb}{0.5,0.5,0.5}
\definecolor{mauve}{rgb}{0.58,0,0.82}
\lstset{frame=tb,
    language=Python,
    aboveskip=3mm,
    belowskip=3mm,
    showstringspaces=false,
    columns=flexible,
    basicstyle={\small\ttfamily},
    numbers=none,
    numberstyle=\tiny\color{gray},
    keywordstyle=\color{blue},
    commentstyle=\color{dkgreen},
    stringstyle=\color{mauve},
    breaklines=true,
    breakatwhitespace=true,
    tabsize=3
}

\title{Homework Assignment N°2}
\author{AML3\\Thibault Douzon\\Georgios Lioutas}
\date{December 15th, 2018}

\begin{document}
\maketitle

\pagebreak

\tableofcontents

\pagebreak
\section{Exercise 1}
\subsection{Part a}
We want to compute for each sample in the data the responsability 
$\gamma(z_k)$ of each Bernoulli distribution.
\\
What we call responsability is the probability knowing the data occured that model $z_k$ is involved.
We can rewrite it in term of probabilities:
$$
\gamma(z_k) = P(z_k \vert x) = \frac{p(x \vert z_k) p(z_k)}{p(x)}
$$
Where $p(z_k)$ is the prior of each model, $p(x\vert z_k)$ is given by the distribution $z_k$ (here Bernoulli)
and $p(x)$ is the sum over all models inthe mixture of the probability they produced $x$.
\\
For example with the first point:
\\
\begin{itemize}
    \item The prior is given by our assumptions: $p(z_k) = \left[0.5, 0.5 \right]$,
    
    \item Each model is a Bernoulli distribution, \\hence $p(x\vert z_k) = 
    \left(
        \begin{array}{c}
            x_0+x_1 \\
            x_0
        \end{array}
    \right) {\mu_k}^{x_0}(1-\mu_k)^{x_1}$
    
    \item And the total probability of $x$ is just the marginalization over all models: $p(x) = \sum_{z_k} p(x \vert z_k) p(z_k)$
\end{itemize}
We finally obtain the following values:
$$
p((1,4)) \approx 0.243
$$
and the responsabilities:
$$
\gamma(z_1) = 0.678
$$
$$
\gamma(z_2) = 0.322
$$
We repeat the same procedure for all given inputs and obtain the following results:
\begin{center}
    \begin{tabular}{ |c|c|c| }
        \hline
        $x$ & $\gamma(z_{n1})$ & $\gamma(z_{n2})$ \\
        \hline
        $(1, 4)$ & 0.678 & 0.322 \\
        \hline
        $(3, 2)$ & 0.345 & 0.655 \\
        \hline
        $(4, 1)$ & 0.208 & 0.792 \\
        \hline
        $(2, 3)$ & 0.513 & 0.487\\
        \hline
    \end{tabular}
\end{center}

\subsection{Part b}
We now want to improve our mixture of model in order to better explain the data we saw.
We must compute a new value for the prior and the parameter of each Bernoulli distribution.
\\
Using the responsabities we just computed in the previous step, the EM algorithm gives us 
formulas to updates thoses values:
\\
The new prior of model $z_k$:
$$
\pi^2_k = \frac{\gamma(z_k)}{\sum_i \gamma(z_i)}
$$
The new proportion of head of model $z_k$:
$$
\mu^2_k = \frac{\gamma(z_k)\mu_k}{\sum_i \gamma(z_i)\mu_i}
$$
But because we now operate in batch after seeing multiple data, we need to modify
those equations by summing over all seen points:
$$
\pi^2_k = \frac{\sum_n\gamma(z_{nk})}{\sum_n \sum_i \gamma(z_{ni})}
$$
$$
\mu^2_k = \frac{\sum_n \gamma(z_{nk})\mu_k}{\sum_n \sum_i \gamma(z_{ni})\mu_i}
$$
With the responsabilities from the previous exercise and the parameters given,
we obtain the following results:
$$
\pi^2 = \left[0.436, 0.564\right]
$$
$$
\mu^2 = \left[0.409, 0.570\right]
$$
This means that we have two models with priors $\pi^2$. Each one of them is still a
Bernoulli model. The first one give a probability of 0.409 to H and the second a probability
of 0.570 to H.

\section{Exercise 2}
\subsection{Part a}
\end{document}
