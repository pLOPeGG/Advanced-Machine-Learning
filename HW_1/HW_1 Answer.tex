\documentclass[a4paper, 10pt]{article}

\usepackage[english]{babel}
\usepackage[T1]{fontenc}
\usepackage[utf8]{inputenc}
\usepackage{textcomp}
\setlength{\marginparwidth}{2cm}

\usepackage{comment}
\usepackage{todonotes}

\usepackage{amsmath}
\usepackage{amssymb}
\usepackage{latexsym}
\usepackage{bm}

\usepackage{enumitem}
\usepackage{array}
\setlength\extrarowheight{5pt}

\usepackage{xcolor}
\usepackage{graphicx}
\graphicspath{ {./img/} }

\newcommand\scalemath[2]{\scalebox{#1}{\mbox{\ensuremath{\displaystyle #2}}}}

\usepackage{hyperref}
\usepackage{listings}
\usepackage{color}
\definecolor{dkgreen}{rgb}{0,0.6,0}
\definecolor{gray}{rgb}{0.5,0.5,0.5}
\definecolor{mauve}{rgb}{0.58,0,0.82}
\lstset{frame=tb,
    language=Python,
    aboveskip=3mm,
    belowskip=3mm,
    showstringspaces=false,
    columns=flexible,
    basicstyle={\small\ttfamily},
    numbers=none,
    numberstyle=\tiny\color{gray},
    keywordstyle=\color{blue},
    commentstyle=\color{dkgreen},
    stringstyle=\color{mauve},
    breaklines=true,
    breakatwhitespace=true,
    tabsize=3
}

\title{Homework Assignment N°1}
\author{AML3\\Thibault Douzon\\Georgios Lioutas}
\date{November 30th, 2018}

\begin{document}
\maketitle

\pagebreak

\tableofcontents

\pagebreak
\section{Exercise 1}
\subsection{Part a}
From the topology of the graph, we deduce that:
$$
P(S) = \sum_c P(D\vert C=c) \sum_d P(I\vert D=d) \sum_i P(S\vert I=i) \sum_g P(G=g\vert D=d, I=i)
$$
Because $P(G\vert D=d, I=i)$ is a probabitlity distribution, $\sum_g P(G\vert D=d, I=i) = 1$.
Our final formula is the following:
$$
P(S) = \sum_c P(D\vert C=c) \sum_d P(I\vert D=d) \sum_i P(S\vert I=i)
$$
$$
P(S) = \left[s_0, s_1\right] = \left[0.516, 0.484\right]
$$
\subsection{Part b}
From the graph we know that
$$
P(G, I=i_0) = \sum_c P(D\vert C=c) \sum_d P(I\vert D=d) P(G\vert D=d, I=i0) \sum_s P(S\vert I=i_0) 
$$
Once again, $P(S\vert I=i_0)$ is a probability distribution, then
$$
P(G, I=i_0) = \sum_c P(D\vert C=c) \sum_d P(I\vert D=d) P(G\vert D=d, I=i0)
$$
$$
P(G, I=i_0) = \left[p(g_0\vert i_0), p(g_1\vert i_0), p(g_2\vert i_0)\right] = \left[0.1384, 0.2272, 0.2024\right]
$$
Because it is not a probability distribution, its sum is not 1. From Bayes rule we know that by dividing by 
$P(I=i_0)$ we would obtain $P(G\vert I=i_0)$ which is a distribution.
$$
P(G\vert I=i_0) = \frac{P(G,I=i_0)}{P(I=i_0)} = \frac{P(G,I=i_0)}{\sum_g P(G,I=i_0)} = \frac{P(G,I=i_0)}{0.568}
$$
$$
P(G\vert I=i_0) =\left[0.244, 0.4, 0.356\right]
$$
Which adds up to 1 accordingly.
\subsection{Part c}
Same process as before:
$$
P(S, G=g_0) = \sum_c P(D\vert C=c) \sum_d P(I\vert D=d) \sum_i P(G=g_0\vert D=d, I=i) P(S\vert I=i)
$$
$$
P(S, G=g_0) = \left[p(s_0\vert g_0), p(s_1\vert g_0)\right] = \left[0.245, 0.148\right]
$$
Because it is not a probability distribution, its sum is not 1. From Bayes rule we know that by dividing by 
$P(G=g_0)$ we would obtain $P(S\vert G=g_0)$ which is a distribution.
$$
P(S\vert G=g_0) = \frac{P(S,G=g_0)}{P(G=g_0)} = \frac{P(S,G=g_0)}{\sum_s P(S,G=g_0)} = \frac{P(S,G=g_0)}{0.393}
$$
$$
P(S\vert G=g_0) =\left[0.624,0.376\right]
$$
Which adds up to 1 accordingly.
\end{document}
